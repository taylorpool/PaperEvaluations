\documentclass{article}
\usepackage{hyperref}

\title{Parallel Tracking and Mapping for Small AR Workspaces}

\author{
    Georg Klein \and
    David Murray
}

\begin{document}
\maketitle

link: \href{https://ieeexplore.ieee.org/stamp/stamp.jsp?arnumber=4538852}{link} 

\section{What Problem is the Paper Addressing?}
Perform AR tracking and map building without any prior map on which to build.

\section{What is the Proposed Solution?}
Separate tracking and mapping into two different threads.
Perform global mapping when possible, but other wise perform local tracking.
Use a four-level pyramid to support the ability to zoom in and out of the camera.
Use FAST corner detector to find features to use for localization and mapping.

\section{What are the Assumptions the Solution Depends on (Explicit and Implicit)?}
\begin{enumerate}
    \item The environment on which we are tracking is small
    \item The environment is static
    \item The computer has at least two threads to use for the process
\end{enumerate}

\section{What is Novel About the Paper?}
The paper separates tracking from mapping. This enables tracking to not 
become fouled up by data association.

The mapping thread does not use features from every single frame. Rather,
the thread utilizes certain keyframes for processing.

\section{Does the Evaluation Included in the Paper Validate/Verify the Claims?}
The paper validates its claim by introducing two augmented reality tests
These tests have low latency due to the stationary nature of their world.
\end{document}